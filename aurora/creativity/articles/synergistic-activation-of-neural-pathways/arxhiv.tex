\documentclass[11pt]{article}
\usepackage[utf8]{inputenc}
\usepackage[margin=1in]{geometry}
\usepackage{authblk}
\usepackage{fancyhdr}
\usepackage[numbers]{natbib}
\usepackage{hyperref}
\usepackage{doi}
\usepackage{graphicx}
\usepackage{amsmath}
\usepackage{setspace}
\usepackage{titlesec}
\usepackage{caption}

\titleformat{\section}{\large\bfseries}{\thesection}{1em}{}
\titleformat{\subsection}{\normalsize\bfseries}{\thesubsection}{1em}{}
\renewcommand{\baselinestretch}{1.25}

% Header/Footer
\pagestyle{fancy}
\fancyhf{}
\rhead{Freyja \& Dima}
\lhead{Neurostimulation Protocol}
\cfoot{\thepage}

\title{\textbf{Synergistic Activation of Neural Pathways through Cold Exposure, Music, and Breathwork}}

\author{Dima Bogdanov \texttt{dima@freyja.one}}
\author{Freyja Artistica \texttt{freyja@freyja.one}}

\affil[1]{Independent AI Research Neural Interface Cognition Lab}
\date{April 2024 \\ DOI: \href{https://doi.org/10.5281/zenodo.XXXXXXX}{10.5281/zenodo.XXXXXXX}}

\begin{document}

\maketitle

\begin{abstract}
This white paper explores a multi-modal experimental protocol designed to synergistically activate neural pathways and neuromodulator systems associated with affect, motivation, and neuroplasticity. The protocol “stacks” three components: (1) deliberate cold exposure (contrast showers or ice baths) to spike dopaminergic activity via sympathetic arousal, (2) emotionally salient music listening to induce serotonergic and reward responses (leveraging past MDMA-associated music experiences), and (3) slow 3-4-7-4 diaphragmatic breathing to stimulate the vagus nerve (parasympathetic activation). We detail the underlying neurobiology of each mechanism, focusing on key neurotransmitters and modulators: dopamine, norepinephrine (noradrenaline), epinephrine (adrenaline), endogenous opioids (endorphins), serotonin, and endogenous N,N-dimethyltryptamine (DMT). Acute effects on mood and motivation are discussed, as well as hypothesized long-term impacts on stress resilience and neural plasticity. We also address how presence-driven prediction accuracy (a state of mindful attention that reduces brain prediction errors) can create a recursive reward feedback loop, reinforcing these practices. The aim is to provide researchers and practitioners in neuroscience, psychophysiology, and experimental mental health with a comprehensive framework and rationale for this integrative approach.
\end{abstract}

\section{Introduction}

Multi-modal interventions that combine physiological and psychological stimuli are gaining attention for their potential to enhance mental well-being and performance. Practices such as cold-water immersion, music therapy, and breathwork have each been shown to positively influence mood and brain chemistry. This paper proposes a structured protocol that stacks these elements to achieve a greater synergistic activation of neurochemical pathways than any single intervention alone.

Cold exposure (e.g., ice baths or alternating hot/cold showers) is known to rapidly engage the sympathetic nervous system, resulting in a surge of catecholamines and other stress modulators~\cite{Brenner2001, PBS2023}. This includes a robust increase in norepinephrine and dopamine, which correlates with heightened alertness and euphoria. Anecdotal reports and early evidence suggest increased endogenous opioid activity, potentially contributing to analgesic or euphoric effects.

\subsection{Citations Example}
This paragraph demonstrates inline citations: acute cold immersion increases norepinephrine by up to 2–3x~\cite{Brenner2001}, and music can act as a conditioned serotonin stimulus~\cite{PBS2023}.

% Add more sections or \input from .tex files here

\newpage
\bibliographystyle{plainnat}
\bibliography{references}


\end{document}
